% ---- chapters/resources.tex ----
\chapter{Recursos descargables}
\label{ch:resources}

Esta sección recopila enlaces y recursos descargables útiles para replicar y mejorar la CNC descrita en este manual. Agrupa modelos 3D, firmware, software y materiales de referencia.

\section*{Modelos 3D (STL)}
\addcontentsline{toc}{section}{Modelos 3D (STL)}
Se recomienda almacenar los archivos STL en la carpeta `assets/stl/` o enlazar el repositorio original. Ejemplos:
\begin{itemize}
  \item Thingiverse — CNC 3D printed: \href{https://www.thingiverse.com/thing:3004773}{https://www.thingiverse.com/thing:3004773}
\end{itemize}

\section*{Firmware}
\addcontentsline{toc}{section}{Firmware}
Firmwares y controladores:
\begin{itemize}
  \item GRBL (firmware para Arduino): \href{https://github.com/grbl/grbl}{https://github.com/grbl/grbl}
\end{itemize}

\section*{Software}
\addcontentsline{toc}{section}{Software}
Herramientas principales para el flujo de trabajo:
\begin{itemize}
  \item KiCad — diseño de PCB: \href{https://www.kicad.org/}{https://www.kicad.org/}
  \item FlatCAM — generación de G-code para PCBs: \href{http://flatcam.org/}{http://flatcam.org/}
  \item CNC.js — interfaz para controlar la CNC: \href{https://cnc.js.org/}{https://cnc.js.org/}
\end{itemize}

\section*{Archivos de ejemplo}
\addcontentsline{toc}{section}{Archivos de ejemplo}
Coloca aquí ejemplos de proyectos (KiCad PCB, FlatCAM project, G-code) en la carpeta `examples/` con subcarpetas por proyecto.

\begin{notebox}
Consejo: usa un archivo `assets/README.md` en la carpeta `assets/` para documentar la procedencia y la licencia de cada recurso descargado.
\end{notebox}

% ---- end of chapters/resources.tex ----